\subsection*{Purpose}
Today, multiple car manufacturers are developing self-driving vehicles (SDVs) \cite{tesla:autopilot}\cite{waymo:home}\cite{honda:autonomous}.
SDVs come in many variants,
and are often classified using the SAE International Levels of Automation \cite{sae:levels}.
Today's cars commonly include simple driver assistants like Cruise Control \cite{wikipedia:acc},
which corresponds to level 1.
At level 2, the system is able to control both steering and acceleration simultaneously.
Several manufacturers are currently working on producing level 3 capable SDVs \cite{honda-level3, mercedes-level3},
which is characterized by the ability for the driver pay attention to other activities,
for instance watching a movie.

But with higher levels comes greater risk,
due to the reduction of human oversight.
In one survey concerning SDVs,
Boston Consulting Group report that "Concerns About the Safety of SDVs Are a Significant Hurdle" for adoption \cite{bcg:survey}.
Another study shows that people generally demand safer-than-human driving
before they trust SDVs on public infrastructure \cite{NEES201961}.
% Thus, to unlock the full market potential of SDVs, manufacturers have to convince the general public of SDVs' safety.

Unfortunately, these strict criteria are not yet met.
Collision report data from SDVs in California reveal a crash rate that is three to five times that of human drivers \cite{10.1115/1.4051779}.
As the authors note, these SDVs even have the benefit of only driving in the ideal weather conditions of sunny California.
Since sensor data quality can be significantly degraded by rainy or snowy conditions \cite{goberville2020analysis},
it is fair to assume that the relative collision rate would be even worse in these conditions.

Thus, motivated by the current lack of sufficient safety in SDVs,
the purpose of this research is to evaluate and improve the safety of SDVs.
We choose to focus on improving safety in harsh winter conditions
to offset the imbalance of current research in clear conditions.
Specifically,
we intend to evaluate and compare the deep learning-based autonomous driver TransFuser \cite{transfuser:pami}
in both snowy and clear conditions in the CARLA simulator \cite{Dosovitskiy17}.
We modify TransFuser to recognize the current weather conditions
and automatically apply weather-dependent safety rules to improve it's safety score.

With this objective in mind,
we propose the following research questions:
\begin{itemize}
    \item \textbf{RQ 1.1} How does the original TransFuser (model \textbf{A}) perform when driving in adverse weather conditions?
    \item \textbf{RQ 1.2} How does a re-trained TransFuser (model \textbf{B}) perform after being trained on data containing adverse weather conditions?
    \item \textbf{RQ 2.1} Can a modified TransFuser trained to predict weather conditions (model \textbf{C}) accurately recognize which conditions it is currently driving in?
    \item \textbf{RQ 2.2} Can weather-specific safety constraints improve the safety of the agent when driving in adverse weather conditions?
\end{itemize}
