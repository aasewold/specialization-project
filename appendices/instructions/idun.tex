\section{Running CARLA on Idun}
\label{app:instructions:idun}

This section details how to prepare and execute CARLA and a client on the Idun HPC cluster.
As previously noted,
Idun provides the Singularity container runtime instead of Docker.
Luckily, this is not a major issue,
since Docker images can easily be converted to Singularity images.
In the following steps we prepare and execute Singularity images for both CARLA and a simple client.


\subsection{CARLA}

To make a Singularity image for CARLA,
use the \texttt{build} command:
\begin{lstlisting}
singularity build carla.sif docker://carlasim/carla:0.9.13
\end{lstlisting}

This can then be executed using the following command:
\begin{lstlisting}
singularity exec --nv carla.sif /home/carla/CarlaUE4.sh -RenderOffScreen
\end{lstlisting}
Note the \texttt{--nv} flag which makes NVIDIA GPUs available in the container,
as well as \texttt{-RenderOffScreen} which prevents the CARLA simulator from attempting to open any GUI windows.
Optionally,
append \texttt{-carla-rpc-port=65001}
to cause CARLA to be available on port 65001 instead of the default port 2000.
This can be utilized to run several CARLA instances simultaneously using different ports.


\subsection{Client}

To demonstrate how a client (such as TransFuser) might connect,
we utilize a simple client we call Carlo, which is available on GitHub.
The repository includes a Dockerfile to build a Docker image,
however we cannot build this image on Idun directly due to the lack of Docker.
Therefore, we build the Singularity image locally
before transferring it to Idun.

Begin by cloning the repository:
\begin{lstlisting}
git clone git@github.com:aasewold/carlo.git
\end{lstlisting}

Enter the directory and build the Docker image:
\begin{lstlisting}
cd carlo
docker buildx build . -t carlo:latest
\end{lstlisting}

Once this is completed,
convert the image to a Singularity image:
\begin{lstlisting}
singularity build carlo.sif docker-daemon://carlo:latest
\end{lstlisting}

Finally, transfer this file to Idun using your preferred tool:
\begin{lstlisting}
rsync -zhP carlo.sif idun-login1.hpc.ntnu.no:carlo.sif
\end{lstlisting}

Now, connect to Idun using SSH,
and execute the Carlo container,
specifying the same CARLA port as used previously:
\begin{lstlisting}
export CARLA_PORT=2000
singularity exec --pwd /code carlo.sif python -m src.scripts.idun $PWD/carlo_output
\end{lstlisting}

The script will connect to the CARLA simulator,
spawn some vehicles in autopilot,
and save some camera sensor data to the folder \texttt{carlo\_output}.

A similar approach can be used to build and execute
a Singularity image for TransFuser.


\subsection{Slurm}

What remains is to perform these steps as part of a batch job using Slurm.
An example Slurm file is given below
that spawns 10 workers
each consisting of a CARLA server and a Carlo client,
launched across different nodes and dynamically assigning
port numbers to prevent conflicts.
\lstinputlisting{listings/idun/carla.slurm}
