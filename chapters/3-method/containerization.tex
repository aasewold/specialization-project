\newpage
\section{Containerization}
\label{sec:containerization}
\todo{subsection of 2?}

To run TransFuser in a reproducible and stable environment,
we defined a Docker container \cite{software:docker}
that contains all required dependencies.

The containerization approach enables reliable reproducability
as well as robust deployment regardless of the target environment.
Our main motivation for this approach is the simplified deployment
required to train and evaluate TransFuser.

Note that we do not use the conda \cite{software:conda} environment
as provided by the TransFuser authors,
but instead install required packages
using the more traditional \texttt{pip} tool \cite{software:pip}.

However, the Docker container runtime is not well-suited for all environments.
Specifically, for HPC clusters such as Idun,
the Singularity container runtime \cite{software:singularity} is often preferred \cite{princeton:singularity}.
This is due to Singularity preventing users from gaining root access, \todo{cite downsides of root access?}
and its immutable container images that simplify concurrent access from multiple compute nodes \todo{cite something?}.

Therefore, as preparations to utilize the Idun cluster for future experiments,
we have produced tooling to generate Singularity images for both the CARLA simulator and TransFuser,
as well as a Slurm script to run batch jobs using these containers on Idun.

Specifically, the Slurm script is capable of launching an arbitrary number of jobs,
where each job consists of an instance of the CARLA simulator and a connected client, for instance an expert agent, TransFuser, or a reinforcement-learning agent.
This makes it possible to perform tasks such as
dataset generation,
TransFuser training and evaluation,
and reinforcement-learning algorithms such as Direct Policy Learning which require an interactive environment.
